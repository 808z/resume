\documentclass[letterpaper,10pt]{article}
%%%%%%%%%%%%%%%%%%%%%%%
%% BEGIN_FILE: mteck.sty
%% NOTE: Everything between here and END_FILE can
%% be relocated to "mteck.sty" and then included with:
\usepackage{mteck}

%% END_FILE: mteck.sty
%%%%%%%%%%%%%%%%%%%%%%

%===================%
% Jeremy's Resume %
%===================%
\begin{document}

  %---------%
  % Heading %
  %---------%

  \documentTitle{Jeremy Galarza}{
    %\href{tel:9156037573}{
      %\raisebox{-0.05\height} \faPhone\ 915-603-7573} ~ | ~
    \href{mailto:jgswe@icloud.com}{
      \raisebox{-0.15\height} \faEnvelope\ jgswe@icloud.com} ~ | ~
    \href{https://www.linkedin.com/in/jeremygalarza}{
      \raisebox{-0.15\height} \faLinkedin\ linkedin.com/in/jeremygalarza}  ~ | ~
    \href{https://www.github.com/808z}{
      \raisebox{-0.15\height} \faGithub\ github.com/808z}
  }

  %-----------%
  % Education %
  %-----------%

\vspace{0pt}
  \section{Education}
  \headingBf{The University of Texas at El Paso \hspace{7.5cm}Expected Graduation Date: Dec. 2026}{}
  \headingIt{B.S in Computer Science, Concentration in Software Engineering $|$ Minor in Mathematics \hspace{7.9cm}}{}
  \vspace{3pt}
  \headingIt{\textbf{Relevant Courses}: Data Structures \& Algorithms, Digital Systems Design, Automata/Computabilty/Formal Languages,
  \\\hspace{0.3cm}Discrete Mathematics, Matrix Algebra, Probability and Statistics}{}
  \vspace{-2pt}
  
  %--------%
  % Skills %
  %--------%
\vspace{0pt}
  \section{Technical Skills}
    \headingIt{\textbf{Languages}: Java, Python, HTML/CSS, JavaScript, TypeScript, Swift, LaTeX, Verilog HDL}{}
    \vspace{1pt}
    \headingIt{\textbf{Technologies}: Artificial Intelligence, Machine Learning, Git, Django, VS Code API, Bash, Linux, QEMU}{}
    \vspace{2pt}
    \headingIt{\textbf{Certifications}: Microsoft Office (MOS), Adobe Creative Suite }{}
  
  %------------%
  % Experience %
  %------------%

\vspace{0pt}
  \section{Experience}

\headingBf{\large{Tech Frontier}}{April - May 2025}
\headingIt{AI Developer Pathway Alumni}{}
\begin{resume_list}
  \item Acquired a strong understanding of fundamental Artificial Intelligence concepts such as \textbf{neural networks}, \textbf{machine learning}, and \textbf{natural language processing}.
  \vspace{1pt}
  \item Leveraged several capabilities of Artificial Intelligence in Python, including \textbf{TensorFlow}, \textbf{Pytorch}, and \textbf{Pomegranate}.
  \vspace{1pt}
  \item Independently developing an iOS application that will demonstrate the \textbf{real world utility} of these concepts to provide \textbf{tangible user benefits}.
  \vspace{2pt}
\end{resume_list}

\headingBf{\large{STTE AI Hackathon}}{April 2025}
\begin{resume_list}
  \item Developed an AI-powered product for the education category in collaboration with a team of four. (SchoolFlow)
  \vspace{1pt}
  \item Applied skills in data analysis, frontend development, AI tools, and implemented an \textbf{SQL database backend}.
  \vspace{-2pt}
\end{resume_list}

  %------------%
  %  Projects  %
  %------------%

\vspace{0pt}
\section{\large{Projects}}

\vspace{1pt}
\headingBf{\large{Summit - AI Outdoors Companion}}{May 2025}
\begin{resume_list}
  \vspace{1pt}
    \item Independently developed a proof-of-concept iOS app that leverages \textbf{on-device AI models} to provide the user with real-time guidance during outdoor activities.
    \vspace{1pt}
    \item Utilizes \textbf{MLC-LLM} in tandem with the \textbf{iOS Speech framework} for real-time speech recognition for user assistance.
    \vspace{1pt}
    \item Integrates \textbf{CoreML} and the \textbf{Vision framework} to provide contextually-aware assistance using the camera feed.
\end{resume_list}

\vspace{0pt}
\headingBf{\large{SchoolFlow}}{April 2025}
\begin{resume_list}
  \vspace{2pt}
    \item Designed an AI-powered platform for school districts to gain actionable insights by correlating real financial data with academic success metrics, visualized through graphs and dashboards.
    \vspace{1pt}
    \item Implemented the website frontend and developed an AI model trained on the district's dataset.
    \item Trained an AI model on the school district's dataset, which was used in conjunction with a team-built \textbf{SQL database} to produce the platform's analytical outputs.
\end{resume_list}

\vspace{0pt}
\headingBf{\large{Bayesian Network Model}}{April 2025}
\begin{resume_list}
  \vspace{2pt}
    \item Used a \textbf{Pomegranate} and \textbf{Pytorch} to create a model that calculates the exact probabilities of different real life events based on user input.
    \vspace{1pt}
    \item Modified the \textbf{knowledge base} of the model to simulate different scenarios that the average person would go through.
\end{resume_list}

\vspace{0pt}
\headingBf{\large{Nim Game Using Machine Learning}}{August 2025}
\begin{resume_list}
  \item Worked on a game in Python that utilized \textbf{machine learning}.
  \vspace{1pt}
  \item Utilized a \textbf{Q-learning algorithm} for the AI opponent which learned based on a specified amount of games, as well as games it has played with the user.
  \vspace{1pt}
  \item Generates a visual output of the game's results using the \textbf{PIL} library.
\end{resume_list}

\vspace{0pt}
\headingBf{\large{HAWK RTL Design for Pedestrian Crosswalk}}{Nov. 2024}
\begin{resume_list}
  \item Recreated a \textbf{High-intensity Activated Crosswalk} design in \textbf{Xillinx Vivado} from UTEP's specifications.
  \vspace{1pt}
  \item Implemented this design using \textbf{Verilog HDL} as an \textbf{Algorithmic State Machine}.
\end{resume_list}

\vspace{0pt}
\headingBf{\large{Wordle}}{Nov. 2023}
\begin{resume_list}
  \item Recreated the popular game New York Times game Wordle using \textbf{Java}.
  \vspace{1pt}
  \item Utilized \textbf{file reading} methods to customize word data.
\end{resume_list}

\end{document}